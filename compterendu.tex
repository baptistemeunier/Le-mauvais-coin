\documentclass{book}
\usepackage[latin1]{inputenc}
\usepackage[T1]{fontenc}
%\usepackage[francais]{babel}
\usepackage{url} % Pour écrire des adresses cliquables.
%\usepackage{lmodern} % Pour changer le pack de police.
\usepackage[top=5cm, bottom=5cm, left=6cm, right=3cm]{geometry} % Les marges.
\title{Projet Base de données}
\author{\textsc{Meunier} - \textsc{Baptiste}}
\date{\today} % Pour mettre la date du jour, tapez \today.
\begin{document}
 
\maketitle % Page de garde.

    
\frontmatter
 
 \section{Principe de fonctionnement}

Exemple avec l'URL /voir-annonce-4

1 - L'utilisateur tombe sur un premier fichier .htaccess à la racine du site.
Ce fichier .htaccess réecrie L'URL en /public/voir-annonce-4

2 - L'utilisateur arrive sur un nouveau fichier .htaccess dans le dossier public.
Ce fichier .htaccess réecrie L'URL en /public/index.php/voir-annonce-4

3 - Puis il exécute le fichier index.php.
Le ficher index.php definie les constantes puis charge l'autoloader et le Dispatcher.

4 - Le Dispatcher crée l'objet Request (Qui va contenir toute les infos de la requete ex : URL, variable GET et POST).
Il va ensuite, grace à la méthode findRoute, parcourir toutes les routes de la classe Config jusqu'à trouver celle qui correspond. (Si aucune ne correspond il revoie la page d'erreur 404). 

5 - Un fois la route trouvée. On cherche le Controller, l'Action et les Paramètres si besoin.
Dans notre exemple on à besoin du Controller (Classe) AnnoncesController, de l'Action (Méthode) viewAction, et d'un paramètre Id (ici 4).

6 - Le controller est chargé puis la méthode est appelle. Tout le code PHP néssessaire à la page est effectué

7 - La méthode retourne une vue avec les variables nécessaire.
Cette vue est affichée sur le navigateur. L'utilisateur peut en lire le contenu. 
\end{document}
